\section{Introduction}

Dans un contexte de classification non-supervisée, l'analyse en composante principale un modèle qui permet de réduire la dimensionalité des données. On peut alors interpréter ces composantes et lorsqu'une projection est fait sur deux dimensions, on peut utiliser le résultat pour faire du classement. Par contre, l'analyse en composantes principales correspond à la meilleure projection linéaire des données. Alors, si la relation entre les variables est non-linéaire ou si elle contient des interactions, l'ACP ne peut pas capturer ces effets lors de la décomposition en valeurs et vecteurs propres. Une alternative à la décomposition de la matrice de corrélations des données est la matrice de corrélation d'une fonction des données, qu'on appelé le noyau. 

