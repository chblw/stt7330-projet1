\section{ACP avec noyau}

Dans le contexte de l'ACP, on trouvait les valeurs et vecteurs propres qui correspondaient à la matrice de covariance. Si les données sont centrées, c.-à-d. $\sum_{i = 1}^{N}x_i = 0$, on a 

\begin{equation*}
C = Var(X) = \frac{1}{N} \sum_{i = 1}^N \textbf{x}_i \textbf{x}_i^T.
\end{equation*}

\cite{scholkopf1997kernel} proposent de répéter cette analyse sur une transformation des données originales, c.-à-d. trouver les valeurs et vecteurs propres de

\begin{equation}\label{eq:covkernel}
\overline{C} = \frac{1}{N} \sum_{i = 1}^{N} \Phi(\textbf{x}_i)\Phi(\textbf{x}_i)^T.
\end{equation}

Ensuite, en appliquant le "truc du noyau" présenté dans la section (\ref{sec:kernel}), on évite de calculer explicitement les données $\Phi(\textbf{x})$, il suffit de calculer la matrice des noyaux. Comme dans l'ACP, on doit faire une décomposition par valeurs et vecteurs propres de 

\begin{equation}\label{eq:covkernel}
\overline{C} = \frac{1}{N} \sum_{i = 1}^{N} K(\textbf{x}_i, \textbf{x}_i).
\end{equation}

\subsection{Normalisation des données}

Dans le développement de l'ACP, on a appliqué l'hypothèse que les données étaient centrées. Par contre, dans le contexte de projection des données dans l'espace $\mathcal{F}$ et pour profiter du truc du noyau, on ne peut pas calculer $$\tilde{\Phi}(\textbf{x}_i) = \Phi(\textbf{x}_i) - \frac{1}{N}\sum_{i = 1}^{N}\Phi(\textbf{x}_i).$$ La solution à ce problème est présentée dans \cite{scholkopf1998nonlinear}. Soit la matrice $K$, où 
$$K_{ij} = (k(\textbf{x}_i, \textbf{x}_j))_{ij}.$$

Il est possible que $\frac{1}{N} \sum_{i = 1}^{N} \Phi(\textbf{x}) \neq 0$. On doit alors centrer les données selon le développent suivant : 

\begin{align*}
\tilde{k}(\textbf{x}_i, \textbf{x}_j) &= \tilde{\Phi}(\textbf{x}_i)^T\tilde{\Phi}(\textbf{x}_j)\\
&= \left( \Phi(x_i) - \frac{1}{N}\sum_{l = 1}^{N}\Phi(\textbf{x}_l) \right)^T \left(\Phi(x_j) - \frac{1}{N}\sum_{l = 1}^{N}\Phi(\textbf{x}_l)\right)\\
&= k(\textbf{x}_i, \textbf{x}_j) - \frac{1}{N}\sum_{l = 1}^{N}k(\textbf{x}_i, \textbf{x}_l) - \frac{1}{N}\sum_{l = 1}^{N}k(\textbf{x}_j, \textbf{x}_l) + \frac{1}{N^2} \sum_{l, k}^{N}k(\textbf{x}_l, \textbf{x}_k).
\end{align*}

Alors, on peut centrer la matrice $\tilde{K}$ selon

\begin{equation*}
\tilde{K}_{ij} = (K - \mathbbm{1}_NK - K\mathbbm{1}_N + \mathbbm{1}_NK\mathbbm{1}_N)_{ij},
\end{equation*}

où $(\mathbbm{1}_N)_{ij} := \frac{1}{N}$ et éviter de calculer les données $\Phi(\textbf{x})$. 

