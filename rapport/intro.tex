\section{Introduction}


L'objectif de l'analyse par composantes principales est d'obtenir une représentation des données dans un espace plus restreint en conservant la plus grande quantité d'information possible. Plus précisément, on considère les combinaisons linéaires des variables mesurées pour l'espace restreint. Par contre, on observe souvent que la relation entre les données n'est pas linéaire. L'analyse par composantes par noyau généralise ce modèle pour des combinaisons non-linéaires des attributs. Au lieu de faire une décomposition par valeurs et vecteurs propres sur la matrice de covariance des données centrées

$$\Sigma = Var(X) = \frac{1}{N} \sum_{i = 1}^{N}   \textbf{x}_i\textbf{x}_i^{T},$$

on fait une décomposition par valeurs et vecteurs propres sur la matrice de covariance des données projetées sur un nouvel espace d'attributs.
