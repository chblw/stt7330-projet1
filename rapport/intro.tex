\section{Introduction}

Dans la résolution de problèmes reliés à l’étude des données, le statisticien peut être confronté à des jeux de données volumineux ayant un grand nombre d’attributs. Ce genre de jeux de données posent certaines difficultés tels un temps d'exécution élevé, le fléau de la dimension ou encore le surapprentissage. Afin de pallier à ces problèmes, quelques méthodes efficaces et reconnues ont été développées afin de réduire mathématiquement le nombre de variables explicatives d’un jeu de données. Une de ces techniques est l’analyse par composantes principales (ACP). Ce rapport présente une généralisation de la méthode permettant d’y intégrer des éléments de non-linéarité. La motivation de cette généralisation est de permettre de capturer autant de variance tout en réduisant davantage le nombre de composantes principales (CP) utilisées.