\section{Conclusion}

En conclusion, il est clair que l'utilisation des noyaux peut apporter un élément additionnel à l'analyse par composante principale.\\

Modéliser l'information non linéaire contenue dans des données dans des espaces à dimension potentiellement infinis est un atout majeur.
Nous avons vu que dans certains jeux de données, ces relations existent réellement et que l'utilisation de noyau permet de les détecter.\\

Cependant, il n'est pas clair si les noyaux sont des outils mathématiques devant être intégrés à tous les problèmes. 
En effet, les noyaux demandent qu'on ajuste des hyper paramètres, ce qui peut être une lourde tâche. Un autre point faible de 
cette approche est que les composantes principales obtenues grâce à la méthode des noyaux ne permettent pas de reconstruire les 
valeurs. \\

Nous croyons qu'il s'agit d'une méthode de plus à utiliser et que celle-ci devrait être comparée aux autres méthodes
traditionnelles lorsque des travaux de science des données demandent de travailler avec un nombre restreint d'attributs.