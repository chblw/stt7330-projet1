\section{Application pratique}

On applique l'analyse par composantes principales et l'analyse par composantes principales avec noyau sur le jeu de données MNIST. Une donnée représente l'intensité de gris entre 0 et 255 des 728 pixels d'une image 28$\times$ 28 de chiffres entre 0 et 9. Elles ont été récoltées par \cite{lecun1998gradient}. Voici un exemple des données : 

\begin{figure}[H]
	\includegraphics[width=\textwidth]{digits-original}
	\caption{4 premiers exemples de MNIST}
	\label{fig:mnist-original}
\end{figure}

On créer aussi un jeu de données modifié de MNIST, où un applique un bruit gaussien. Voici les mêmes exemples que dans la figure \ref{fig:mnist-original} :

\begin{figure}[H]
	\includegraphics[width=\textwidth]{digits-noisy}
	\caption{4 premiers exemples de MNIST avec bruit gaussien}
\end{figure}

Une analyse par composantes principales est identique à une analyse par composantes principales avec un noyau linéaire. On projette les données selon les deux premières composantes principales. On obtient

\begin{figure}[H]
	\includegraphics[width=\textwidth]{comparaison-lineaire}
\end{figure}

Il n'y a pas de flexibilité à cette méthode. On remplace la matrice variance-covariance par différents noyaux et on présente les projections dans la prochaine figure. 

\begin{figure}[H]
	\includegraphics[width=\textwidth]{comparaison-kernel}
	\caption{Projection des deux premières composantes principales selon différents noyaux.}
\end{figure}

On remarque que le choix du noyau a beaucoup d'importance sur la projection. Les données en rouge représentent le chiffre 1 et la plupart des noyaux peuvent séparer les données. Le noyau polynomial est performant pour segmenter les chiffres 0. 

On applique ensuite l'ACP avec noyau sur les données MNIST bruitées. On obtient 

\begin{figure}[H]
	\includegraphics[width=\textwidth]{comparaison-noisy}
	\caption{Projection des deux premières CPs selon différents noyaux.}
\end{figure}

On remarque que certains noyaux sont moins performants pour capturer l'information que sans le bruit, mais que certains sont capables de trouver les relations non linéaires dans les données bruitées pour retrouver l'information originale. 
